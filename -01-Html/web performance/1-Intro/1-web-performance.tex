- Web performance is a term describing how fast a website or web application is perceived to be.

Performance Measurements 
  first contentful paint
  largest contentful paint (LCP)
  first meaningful paint (FMP)
  time to interactive  (TTI)
  max potential first input delay

Why performance Matters 
  websites and applications are becoming more complex 
  Increased complexity means increased file sizes 
  User expectations for fast web services increasing 
  Extreme disparity in internet access and services quality 
  large variance in devices and software used to access the web 

What need to Achieve 
  websites and applications need to be fast and efficient for all users no 
  matter what combination of these conditions they're working under. 

performance optimization Areas
  Reducing overall load time 
    This is done by compressing and minifying all files
    reducing the number of file and other HTTP requests sent back and forth between the server and the user agent
    employing advanced loading and caching techniques 
    and conditionally serving the user with only what they need when they actually need it. 

  Making the site usable as soon as possible 
    by loading critical components first to give the user initial content and functionality 
    then deferring less important features for later using lazy loading to request 
    display content only when the user gets to or interacts with it.
    by pre-loading features, the user is likely to interact with next. 

  Smoothness and interactivity 
    this is about improving the perceived performance of a site through skeleton interfaces, 
    visual loaders and clear indication that something is happening and things are going to work soon.

  Performance measurements 
    We use tools and metrics to monitor performance and validate up my station efforts.

Practical web performance strategy 
Do what you can, and realize not every performance optimization will fit your situation and needs.

  

  