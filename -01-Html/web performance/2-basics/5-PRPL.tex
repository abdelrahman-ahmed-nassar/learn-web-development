- [Instructor] There's a useful mnemonic device you can use to remember how to 
achieve the best possible performance for your website or application. PRPL, or 
P-R-P-L, referring to the PRPL pattern. This is an acronym that stands for push or
preload the most important resources, render the initial route as soon as possible, 
pre-cache remaining assets, and lazy load other routes and non-critical assets. 
This pattern has become the industry standard benchmark for web performance for two
main reasons. It was developed by the leading industry experts on performance and 
it just makes sense. If we push or preload important resources to the browser using
server push for the initial load and service workers in the next round, the 
application will run faster. If we render the initial route as soon as possible by 
serving the browser with critical CSS and JavaScript, the perceived performance of
the application will be improved. If we pre-cache remaining assets so they are 
available when the browser needs them, the application performance will be 
dramatically improved. And finally, if we lazy load all non-critical assets so 
they only load when they are actually needed, we reduce the time to initial load
and save the visitor from wasting their bandwidth on assets they will never use.
As you make your way through this course, keep the PRPL pattern in mind 
almost everything that we're going to cover fits into one of these steps, push
our preload, render, pre-cache, and lazy load.


THE PRPL pattern 
  Push (or preload) the most important resources 

  render the initial route as soon as possible 

  pre-cache remaining assets 

  laze load other routes and non-critical assets 