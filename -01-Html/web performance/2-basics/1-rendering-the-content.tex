How browsers render content - [Instructor] To understand performance optimization, you first need a solid
understanding of how typing something into the address bar of a browser results 
in the page being rendered in the viewport. So let's follow our requests to some
site.com all the way from the initial request to the final results. It all 
starts with the browser sending a request for some site.com to its ISP, that's 
the internet service provider for the current internet connection the browser is
connected to, that could be your regular internet provider, if you're in your 
house or an office, or it could be your cell phone provider, if you're using 
your mobile phone, the ISP then sends the request immediately to a DNS domain 
name service. This is effectively a phone book for the web, associating the 
host name or URL. So Universal Resource Identifier in our case, some site.com 
with an IP address the actual location of that service. This by the way, is how 
you can move a domain name from one server to another, and why it takes a while 
to do so. You're really just changing the entry in the domain name service to 
this site with this name now lives up this new address. And the reason why it 
takes a while is because the domain name service has to be updated. And then 
that update has to be distributed all across the web. Anyway, this DNS lookup
is done for each unique hostname. So if the site you're requesting is using
externally hosted fonts, or JavaScript libraries, or images, or videos or 
services, this DNS lookup happens for each of those different services. 
Anytime there's a new domain name, a new DNS lookup have to take effect.
This is the first major performance bottleneck. To do away with some of this 
performance overhead, the domain name to IP address association will probably be 
cached at numerous different steps, your ISP will cached as information, it will 
also likely be cached in your router and on your computer. That way when you send
a request to the same domain you requested before, instead of having to go through 
the whole DNS lookup again, we're just pulling a cache from somewhere closer to the 
computer, but that also means if the DNS has changed in the meantime, you'll get an 
incorrect address pointing and things won't work as expected. Once the IP address is 
established, the browser and server now perform what's called a TCP handshake, where 
they exchange identity keys and other information to establish a temporary 
and working relationship. It is quite literally, hello, I am browser. Hi, I'm 
server. Hi server, I have a request. Okay, go ahead browser I am ready to serve. 
This is also where the type of connection is determined this is there's a regular 
HTTP connection or is it an encrypted HTTPS connection? If the latter, encryption 
keys are exchanged and if both the browser and the server support it, the 
transaction is updated from HTTP 1.1 to HTTP two, which provides substantial 
performance enhancements, We now have a connection and everything is ready to go. 
At this point, the browser sends an HTTP GET request for the resource it's looking 
for. This initial GET request will be for whatever the default file on the server 
location is, typically index.html or index.php or index.js or something similar to
that. The time it takes for the browser to finally receive the first byte of the 
actual page it's looking for, is measured in time to first byte or TTFB, the first
true performance measurement. The first piece of data called the packet that the
browser receives is always 14 kilobytes, then the packet size doubles with every
new transfer. That means if you want something to happen right away, you need 
to cram it into those first 14 kilobytes. The browser now gets a file an HTML 
document, and it starts reading it from top to bottom and then parsing that 
data. This means the HTML is turned into a DOM tree, the CSS is turned into a 
CSSOM tree and object model for the CSS of the page, which makes it possible 
for the browser to render the CSS for JavaScript to interact with it. And ask 
the document is parsed, the browser also loads in any external assets as they 
encountered. That means anytime it encounters a new CSS file, or reference to 
anything else, it'll send a new request, the server responds by sending the 
request back, then it gets placed into the system, and the browser starts 
rendering that as well. In the case of JavaScript, though, the browser stops 
everything else and waits for the file to be fully downloaded. Why? Because 
there's a good chance of JavaScript wants to make changes to either the DOM or
the CSSOM or both. This is what's known as render blocking, whatever
rendering was happening, stops and is literally blocked for as long as the browser 
is waiting for the JavaScript to be fully loaded and then fully executed. Once all 
of this parsing is done, the rendering can begin in earnest and here the browser 
the DOM and CSSOM to style, layout, paint and composite the document in the 
viewport. This is the process that actually creates what you see in the browser. 
The metric time to first Contentful paint refers to how long it takes for all of 
this to happen and this step to start taking place, as you can see there is a lot 
going on behind the scenes when a website or app is loaded into the browser. And 
this breakdown is a severely simplified version of true events. What's important 
for our purposes is to remember what's actually happening, that way we can identify
bottlenecks and add performance enhancements to get past them as quickly as 
possible.


